\documentclass{article}
\usepackage[margin=1in]{geometry}
\usepackage{graphicx}
\usepackage{a4wide}

\begin{document}
\title{Report --- Juniour Honours Project Submission 1}
\author{Hafeez Abdul-Rehaman, Johannes Weck, Josh Lee, Calum Duff, Tom Harley}
\date{\today}

\maketitle

\section{Requirements Overview}
For our Junior Honours project, we are to design and create a database
for a system which will be used primarily for the sharing and viewing
of various medical and pathological media.

Our database is to be:
\begin{description}
  \item[User-aware] The system should act as a platform for managing
    users and their permissions, with respect to viewing and affecting
    projects and media.
  \item[Filetype-agnostic] The system should accept files in a
    variety of filetypes and be able to translate between them, in order to
    provide a user with a desirable output filetype.
  \item[Substitutable] The system should follow standardised
    communication protocols, to participate in a framework of modularised
    systems created by the other Junior Honours groups.
\end{description}

\section{Using Scrum}
The Scrum approach facilitated the
planning and running of group meetings. As the year-wide protocol changed
over time, an AGILE working process provided the granular iteration
required to address the frequent design changes.

\subsection{Scrum Board}
Throughout the project a Scrum board has been used to keep track of current tasks
and further items to implement. A whiteboard in the John Honey main computer lab
was used throughout this process. Examples of the board in use can be found in
Appendix A.

The Scrum board made visualizing tasks and progression straightforward.
Using it in meetings helped with determining how much was still to be done.

At times the \emph{In Progress} section of the Scrum board seemed unbalanced;
at these times several people were working on the same large task, and so fewer
notes were placed in this column.

\subsection{Time Management}
Based on how tasks were split, and the difficulty of various tasks, we estimated
durations for each task. Due to the widely-varying estimated completion times,
we decided to distribute our Sprint durations accordingly.
Initially, we divided our total available time with the number of available tasks,
but over time experience from previous work and from previous meetings was helpful in
refining the expected duration of the future tasks.
This, combined with
the late reveal of the specification meant that work wqs done over holidays, greatly 
warping the duration of several Sprints to be longer than anticipated. This is
evident in the burnout graphs provided for the period over December%
\footnote{The burnout graph we generated is available in Appendix B.}.

\subsection{Meetings}
...
\footnote{The minutes for formal meetings are available in Appendix C.}

\subsection{Paired Programming/Teamwork}
Paired Programming was used extensively throughout implementation of basic server
functions, such as writing logic for the public protocol endpoints.

Multiple people working on a single task reduces the total actual time it takes to
complete. This allowed larger tasks that would not normally fit into a regular Sprint
to still be tackled within a single Sprint.

At times the entire team was in labs, working concurrently on separate tasks.

Having everyone in the same place focussed everyone on the task at hand and greatly
improved communication, keeping everyone working on what was important and not dallying.

\subsection{Evaluation of Work Done/To Do}
The Scrum board and regular meetings greatly aided in our evaluation of how much work
had been completed, and how much there was left to do, at the time. Now that we're
used to Scrum, evaluation of further tasks (such as extensions) should be more
realistic, and tasks are more likely to be assigned to Sprints with enough time to
accommodate them, instead of persisting to the next Sprint.

\section{Design Decisions}
\subsection{Database}
Passwords in the database are currently stored totally unencrypted. Initially,
the focus of the team's work was on routing traffic, and for this purpose salting
passwords is unnecessary. Once the server is past development phase, passwords
will be encrypted in some secure fashion.

\subsection{Server Framework}
The protocol designed by the entire year uses a token-based authentication service
similar to and inspired by OAuth2. Using token-based authentication removes the need
for sessions or for cookies while still providing a smooth and secure user experience,
and simplifying client design.
It was decided that implementing OAuth2 in its entirety would be superfluous. OAuth2
relies on the acquisition of tokens from third parties, which would be unsuitable for
a simple system such as this.

\subsection{Testing Framework}
...

\section{Looking Ahead}
\subsection{File Agnosticity}

\subsection{Advanced Telemetry}

\subsection{ACID Distribution}


\end{document}

