\section{Logging}
Server logging has now been formalised: the functionality is imported into
a file and can then be used to log messages at the selection of urgencies
specified in Protocol BE01~\cite{protocol}.

These outputs have many configuration options, which ideally will be
incorporated into a global configuration handler in future development.

\subsection{Standard Logging}
By default, the standard logger logs to three places:
\begin{enumerate}
	\item Colourised, human-readable messages are printed to the console.
		Any error- or critical-level messages are printed through stderr.
	\item More verbose JSON editions of messages are written to a basic
		logfile.
	\item Critical-level messages are additionally written to an extra
		logfile.
\end{enumerate}

\subsection{SQL Query Logging}
A separate logger has been designed for logging the output of
Sequelize~\cite{sequelize}. Since Protocol BE01 does not specify that they
should be retreivable, the output queries can be stored with less associated
metadata. In addition, Sequelize can output the total required time to
process the query, which is stored in the log for future use as profiling
data.

\subsection{Motivation}
Since basic server functionality has been implemented, the errors produced
will likely be caused by difficult-to-debug problems, such as edge-cases or
data races. With a verbose and thorough logging system, these problems
should be much easier to detect and rectify.

In addition, Protocol BE01~\cite{protocol} requires of the backend support
for receiving and storing logs from other services. The introduction of a
well-implemented logging system abstracts this from the common use-case of
logging while keeping powerful functionality of forwarding and fetching logs
available for when it is needed.
