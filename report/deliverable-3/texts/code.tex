\section{Code Quality}
\subsection{Linter}
To ensure code clarity and quality, we decided to use the linter `ts-lint'. This did not only help us create professional-style code but also aided in avoiding bugs early by catching out common errors. The linter was integrated with our editors as well as yarn as to automate its work-flow.

\subsection{Scripts}
Several scripts were created to automate the installation and execution process.

\subsection{Caching}
As we carried out testing of the serving of tileable images, we were finding that there was a significant amount of wasted computation taking place
which was slowing down requests significantly. This was as a result of computing/stitiching and cutting regions of images even as a result of a
repeated request from the frontend even if we had just done and written this to a file.

In order to combat this we added a further database to our backend design called Redis. This database is optimised for caching.
We decided that the best item to cache would be a specific URL request since this contained all the parameters used to figure
out the region of the image to get. The value of the URL key, would be the filepath of the image which matches this URL's parameters on disk.
Therefore if a duplicate request comes in, we can skip all image computation and simply stream the results of the image back to the client instantly.

We planned to push this caching even higher still, and cache every request made to the server. This would be especially beneficial to things requiring access
into supported views etc since file reads can be skipped and any navigation of the other database/filesystem or raw file computation/generation could be skipped.
Regrettably we did not have the time to get this far, so currently only image tile requests are cached.
